\documentclass[journal,12pt,twocolumn]{IEEEtran}

\usepackage{enumitem}
\usepackage{amsmath}
\usepackage{amssymb}
\usepackage{gensymb}
\usepackage{graphicx}
\usepackage{txfonts}         
\usepackage{listings}
\usepackage{lstautogobble}
\usepackage{mathtools}
\usepackage{bm}
\usepackage{hyperref}
\usepackage{polynom}
\usepackage{capt-of}
\usepackage{siunitx}
\newcommand{\solution}{\noindent \textbf{Solution: }}
\newcommand{\mycomment}[1]{}
\providecommand{\pr}[1]{\ensuremath{\Pr\left(#1\right)}}
\providecommand{\brak}[1]{\ensuremath{\left(#1\right)}}
\providecommand{\cbrak}[1]{\ensuremath{\left\{#1\right\}}}
\providecommand{\sbrak}[1]{\ensuremath{\left[#1\right]}}
\providecommand{\mean}[1]{E\left[ #1 \right]}
\providecommand{\var}[1]{\mathrm{Var}\left[ #1 \right]}
\providecommand{\der}[1]{\mathrm{d} #1}
\providecommand{\gauss}[2]{\mathcal{N}\ensuremath{\left(#1,#2\right)}}
\providecommand{\mbf}{\mathbf}
\providecommand{\abs}[1]{\left\vert#1\right\vert}
\providecommand{\norm}[1]{\left\lVert#1\right\rVert}
\providecommand{\z}[1]{{\mathcal{Z}}\{#1\}}
\providecommand{\ztrans}{\overset{\mathcal{Z}}{ \rightleftharpoons}}

\providecommand{\parder}[2]{\frac{\partial}{\partial #2} \brak{#1}}

\let\StandardTheFigure\thefigure
\let\vec\mathbf

\numberwithin{equation}{section}
\renewcommand{\thefigure}{\theenumi}
\renewcommand\thesection{\arabic{section}}

\newcommand{\myvec}[1]{\ensuremath{\begin{pmatrix}#1\end{pmatrix}}}
\newcommand{\mydet}[1]{\ensuremath{\begin{vmatrix}#1\end{vmatrix}}}
\newcommand{\define}{\stackrel{\triangle}{=}}

\DeclareMathOperator*{\argmin}{arg\,min}
\DeclareMathOperator*{\argmax}{arg\,max}

\lstset {
	frame=single, 
	breaklines=true,
	columns=fullflexible,
	autogobble=true
} 

\begin{document}

\title{ Assignment-2 \\ \Large EE3900: Linear Systems and Signal Processing \\ \large Indian Institute of Technology Hyderabad}
\author{Shivanshu \\ \normalsize AI21BTECH11027 \\ \vspace*{20pt} \normalsize 28 September 2022}
\maketitle

\section{Difference Equation}
\begin{enumerate}[label=\thesection.\arabic*]
    \item \textbf{[Question 2.4 from Discrete-Time Signal Processing by Alan V. Oppenheim] : }Consider the linear constant-coefficient difference equation:
          \begin{align}
              y\sbrak{n}-\frac{3}{4}y\sbrak{n-1}+\frac{1}{8}y\sbrak{n-2} & =2x\sbrak{n-1}
          \end{align}
          Determine $y\sbrak{n}$ for $n\ge 0 $ when $x\brak{n}=\delta \sbrak{n}$ and $y\brak{n}=0,n<0$.
\end{enumerate}
\solution By applying Z-transform,
\begin{align}
    {\mathcal {Z}}\{y\sbrak{n}\}-\frac{3}{4}{\mathcal {Z}}\{y\sbrak{n-1}\}+\frac{1}{8}{\mathcal {Z}}\{y\sbrak{n-2}\}=2{\mathcal {Z}}\{x\sbrak{n-1}\}
\end{align}
As,
\begin{align}
    \label{eq:z-trans-prop}
    {\mathcal {Z}}\{x(n-k)\} & = z^{-k}X(z)
\end{align}
\begin{align}
    Y(z)-\frac{3}{4}z^{-1}Y(z)+\frac{1}{8}z^{-2}Y(z) & =2z^{-1}X(z)
\end{align}
As,
\begin{align}
    x\brak{n}                                        & =\delta \sbrak{n}              \\
    {\mathcal {Z}}\{x(n)\}                           & ={\mathcal {Z}}\{\delta(n)\}=1 \\
    Y(z)-\frac{3}{4}z^{-1}Y(z)+\frac{1}{8}z^{-2}Y(z) & =2z^{-1}                       \\
    Y(z)(1-\frac{3}{4}z^{-1}+\frac{1}{8}z^{-2})      & =2z^{-1}
\end{align}
\begin{align}
    Y(z) & =\frac{2z^{-1}}{1-\frac{3}{4}z^{-1}+\frac{1}{8}z^{-2}}                   \\
         & =\frac{-8}{4z-1}+\frac{8}{2z-1}                                          \\
    \label{eq:main}
         & =\frac{4z^{-1}}{1-\frac{1}{2}z^{-1}}-\frac{2z^{-1}}{1-\frac{1}{4}z^{-1}}
\end{align}
\textbf{Case-1:}When ROC is $\abs{z}>\frac{1}{2}$, the signal x(n) is causal. Both terms in \eqref{eq:main} are causal.
As,
\begin{align}
    \label{eq:anti}
    a^{n-1}u(n-1) \ztrans \frac{z^{-1}}{1-az^{-1}} \quad \abs{z} > \abs{a} \\
    y(n)=4\brak{\frac{1}{2}}^{n-1}u(n-1)-2\brak{\frac{1}{4}}^{n-1}u(n-1)   \\
    y(n)=\brak{\frac{1}{2}}^{n-3}u(n-1)-\brak{\frac{1}{4}}^{n-\frac{3}{2}}u(n-1)
\end{align}
\textbf{Case-2:}When ROC is $\frac{1}{4}<\abs{z}<\frac{1}{2}$, ROC is a ring. The signal is two-sided. Thus the pole $p_1=\frac{1}{2}$ provides anti-causal part and $p_2=\frac{1}{4}$ provides causal part.
\begin{align}
    -a^{n-1}u(-n) \ztrans \frac{z^{-1}}{1-az^{-1}} \quad \abs{z} < \abs{a} \\
    y(n)=-4\brak{\frac{1}{2}}^{n-1}u(-n)-2\brak{\frac{1}{4}}^{n-1}u(n-1)   \\
    =\brak{\frac{1}{2}}^{n-3}u(-n)-\brak{\frac{1}{4}}^{n-\frac{3}{2}}u(n-1)
\end{align}
\textbf{Case-3:}When ROC is $\abs{z}<\frac{1}{4}$, the signal x(n) is anti-causal. Both terms in \eqref{eq:main} are anti-causal.\\
Using \eqref{eq:anti},
\begin{align}
    y(n) & =-4\brak{\frac{1}{2}}^{n-1}u(-n)+2\brak{\frac{1}{4}}^{n-1}u(-n)        \\
    y(n) & =\brak{\frac{1}{2}}^{n-3}u(-n)+\brak{\frac{1}{4}}^{n-\frac{3}{2}}u(-n)
\end{align}
\end{document}
